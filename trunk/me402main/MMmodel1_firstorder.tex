Our \gls{first-order model}---a first-order, linear, ordinary differential equation---is 
\begin{equation}
\label{e:first}
\frac{dy(t)}{dt} + \frac{1}{\tau}y(t) = f(t),
\end{equation}
where $f(t)$ is the input (forcing function), $y(t)$ is the output (response function) and $\tau$ is the time constant.  That's it; there is really nothing more to say.  

But of course we do have much more to say.  We should start by emphasizing that this model is just as wrong as the rest of them, but it can be very useful.  There are two reasons that this model is useful:
\begin{enumerate}
\item The time-response of this model is sufficiently similar to that of many physical systems to allow us to use this fictitious model (equation) to predict how an actual system might behave.
\item We know (or we will know shortly) how to solve this equation to calculate the output of the model in response to a variety of inputs.
\end{enumerate}







